% $ABSTRACT_SESSION_Va_5a


\atitle{Binary Progenitors for Long-Duration Gamma-Ray Bursts}

\bigskip

\authors{Philipp Podsiadlowski}

\affiliation{Oxford University}

\bigskip

\noindent In this talk I review recent progress on binary progenitor models for
long-duration gamma-ray bursts, involving tidal spin-up models, merger
models and explosive common-envelope ejection. Most of these models
require a late binary interaction (so-called Case C mass transfer) and
work better at lower metallicity. However, unlike some popular
single-star models, this is not required and may prove to be an
important discriminant. Indeed, this is
consistent with recent evidence that LGRBs do not avoid high-Z host galaxies.  Estimates for the various channels are consistent with an estimated LGRB rate of $\\sim 10^{-5}\\,$yr$^{-1}$ in
a typical galaxy, in particular, if the newly recognized mode of Case
D mass transfer contributes to the LGRB rate.
