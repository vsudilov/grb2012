% $ABSTRACT_SESSION_Va_5a


\atitle{Long gamma-ray bursts from interacting binaries}

\bigskip

\authors{Ross Church [1], Melvyn Davies [1], Andrew Levan [2], Chunglee Kim [3]}

\affiliation{[1] Lund University, [2] University of Warwick, [3] West Virginia University}

\bigskip

\noindent The origin of long-duration gamma-ray bursts in the core-collapse of
rapidly-rotating massive stars is well supported by observations. However, it is difficult to prevent the progenitor's spin being braked by strong winds on the main sequence.  This prevents the post-core-collapse accretion disc forming.  The problem is worsened by recent observations of long gamma-ray bursts in high-metallicity regions.  I will show that, in a close binary, it is possible to replenish the spin of the star's core using tides to extract angular momentum from the binary's orbit.  This mechanism requires the progenitor to be in a close binary with a black hole.  I will show that this scenario naturally predicts features in the accretion history that can be tested using light curves of long gamma-ray bursts.

\index{\tiny{Church, Ross: \textit{Long gamma-ray bursts from interacting binaries}}}
