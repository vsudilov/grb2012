% $POSTERS_SESSION_PVII_p7
P-VII-1

\atitle{Could we expect  Ultra High Energy Cosmic Rays from Centaurus A?}

\bigskip

\authors{Nissim fraija, Magdalena Gonzalez, Sarira Sahu, Antonio Marinelli, Miguel Perez}

\affiliation{Astronomy Institute}

\bigskip

\noindent The Pierre Auger Observatory has associated a few ultra high energy cosmic rays  with the direction of Centaurus A. Its spectral energy distribution or spectrum shows two main peaks, the low energy peak, at an energy of  0.01 eV,  and the high energy peak, about 150 keV. There is also a faint very high energy (E > 100 GeV) gamma-ray emission fully detected by the High Energy Stereoscopic System experiment.  In this work  we describe the entire spectrum, the two main peaks with a Synchrotron/Self-Synchrotron Compton model and, the VHE emission with a hadronic model.  For the hadronic model we consider pgamma and pp interactions. When considering  pp interaction to describe the gamma-spectrum, the obtained number of ultra high energy events are in agreement with Pierre Auger observations.

\index{\tiny{Fraija, Nissim: \textit{Could we expect  Ultra High Energy Cosmic Rays from Centaurus A?}}}
