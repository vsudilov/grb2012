% $ABSTRACT_SESSION_Vc_5c


\atitle{GRB central engine, jet formation and propagation}

\bigskip

\authors{Ehud Nakar}

\affiliation{Tel Aviv University}

\bigskip

\noindent The requirements from a GRB central engine are extreme: ejection of highly luminous, narrowly collimated, relativistic jets. The jet variability time scale is milliseconds, while the engine must be active for seconds to minutes and possibly even hours. I will first discuss the two popular engine models, accreting black-hole and proto-magnetar, and the constraints that they impose on the progenitor system of both long and short GRBs.  I will then focus on the Collapsar model for long GRBs and on the interaction of a relativistic jet with a stellar envelope. I will show that this interaction leaves an observed signature, which is detectable in the GRB samples of all major satellites, and discuss what can be learned from this signature on the typical GRB engine activity time.

\index{\tiny{Nakar, Ehud: \textit{GRB central engine, jet formation and propagation}}}
