% $POSTERS_SESSION_PIII_p3
P-III-3

\atitle{Afterglow emission in the context of an “one-zone” radiation-acceleration model}

\bigskip

\authors{M. Petropoulou [1], A. Mastichiadis [1], T. Piran [2]}

\affiliation{[1] National and Kapodistrian University of Athens, Department of Physics, [2] Racah Institute of Physics, The Hebrew University, Jerusalem}

\bigskip

\noindent In the present work, we focus on the interplay between stochastic acceleration of charged particles and radiation processes in a region of turbulent magnetized plasma, setting the framework for an “one-zone” radiation-acceleration model for GRB afterglows. Specifically, we assume that the particle distribution is isotropic in space and treat in detail the particle propagation in momentum-space. The electron distribution is modified by the acceleration, radiation (synchrotron/SSC) and escape processes.  The magnetic field and the particle injection rate and energy are functions of  time as measured in the co-moving frame of the blast wave.  We numerically solve the time-dependent Fokker-Planck equation and present characteristic examples for the obtained particle and photon spectra.

\index{\tiny{Petropoulou, Maria: \textit{Afterglow emission in the context of an “one-zone” radiation-acceleration model}}}
