% $ABSTRACT_SESSION_IIIc_3c


\atitle{Gamma-ray burst afterglows as probes of the ISM}

\bigskip

\authors{Patricia Schady}

\affiliation{MPE Garching}

\bigskip

\noindent It has long been recognized that the bright and simple afterglow spectra of GRBs make highly effective probes of the ISM within distant, star-forming galaxies. Nevertheless, to fully exploit their potential, several outstanding issues need to be addressed. Most notably, the absorption imprint left on the X-ray afterglow from intervening material implies an order of magnitude more gas than is measured on optical spectra. I will present results from a comprehensive study on the multi-wavelength attenuation of GRB afterglows, and discuss the implied origin of the X-ray absorption excess. I shall also review current research on the extinction properties of host galaxy dust; a topic highly pertinent to optically `dark’ GRBs and to far-infrared/submm observations of host galaxy dust emission

\index{\tiny{Schady, Patricia: \textit{Gamma-ray burst afterglows as probes of the ISM}}}
