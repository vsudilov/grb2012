% $POSTERS_SESSION_PII_p2
P-II-3

\atitle{Search for extended emission in Fermi/GBM GRBs}

\bigskip

\authors{G. Fitzpatrick [1], V. Connaughton [2], S. McBreen [1]}

\affiliation{[1] University College Dublin,  [2] Center for Space Plasma and Aeronomic Research, University of Alabama in Huntsville}

\bigskip

\noindent Gamma-Ray Bursts (GRBs) are characterized at high energies in their prompt emission by impulsive peaks with sharp rises, often highly structured, and easily distinguishable against instrumental backgrounds. The longer-lived afterglow radiation seen at lower energies is much smoother and would be difficult to detect in a background-limited instrument such as the Gamma-ray Burst Monitor (GBM) onboard Fermi. Observations above 100 MeV of this type of long-lived emission from bright GBM detected GRBs by the Fermi Large Area Telescope (LAT) suggest the possibility of extended lower-energy gamma-ray emission. In order to search for such emission in GBM GRBs we have developed a new background estimation tool. We report the results of this search on GRBs from the first 3 years of operation.
