% $POSTERS_SESSION_PII_p2
P-II-32

\atitle{Short X-ray Transients detected by MAXI/GSC}

\bigskip

\authors{N. Kawai [1], T. Toizumi [1], M. Morii [1], R. Usui [1], M. Serino [2], T. Mihara [2], M. Matsuoka [2], M. Sugizaki [2], A. Yoshida [3] and the MAXI Team}

\affiliation{[1] Tokyo Tech, [2] RIKEN, [3] AGU}

\bigskip

\noindent We searched for short X-ray transients in the 15 month data of Monitor of All-sky X-ray Image (MAXI) in the 4.0-10.0 keV energy band.  MAXI scans most of the sky every 92 minutes with a narrow fan-beamed field of view with a typical transit time of 50 seconds.  In this study, we searched for transient sources that are detected in single scans of MAXI/GSC at Galactic latitudes larger than 20 degrees.  As a result, we found 29 transients with the fluctuation probability of less than 10$^{-8}$.
Eight of them coincides with prompt emission of GRBs reported to GCN, and one identified as the afterglow of a reported GRB.
Three transients are positionally coincident with flare stars, and other two with known X-ray sources.  
The other 15 transients remained unidentified.
The Log N-Log S distributi

\index{\tiny{Kawai, Nobuyuki: \textit{Short X-ray Transients detected by MAXI/GSC}}}
