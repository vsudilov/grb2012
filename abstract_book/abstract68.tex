% $POSTERS_SESSION_PII_p2
P-II-11

\atitle{Simulation of high energy emission from gamma-ray bursts and other astronomical sources}

\bigskip

\authors{Houri Ziaeepour}

\affiliation{MPE Garching}

\bigskip

\noindent High energy gamma-ray has been detected from various astronomical sources including gamma-ray bursts (GRBs). Notably, observations in MeV and GeV bands have shown a delay of up to tens of seconds from low energy photons for high energy photons. The origin of this emission is not well understood. After a brief summary of suggested explanations for these observation, I review a model and its simulations that describe various spectral features of GRBs at high energies. Based on these results and behaviour of plasma in relativistic shocks, I explain the origin of the delayed emission by associating it to trapped electrons in the electromagnetic structure at the shock front. The same process can be involved in the acceleration of cosmic ray to very high energies by e.g. shocks in clusters.
