% $ABSTRACT_SESSION_IIa_2a


\atitle{The Fermi Era: Towards a better understanding of the GRB prompt emission}

\bigskip

\authors{sylvain Guiriec}

\affiliation{NASA Goddard Space Flight Center / ORAU}

\bigskip

\noindent Since 2008, prompt gamma-ray observations either by Fermi alone or jointly with Swift have changed our view of Gamma-Ray Bursts (GRBs). While Fermi data reproduce globally the results obtained a decade ago by CGRO from keV to GeV,the new capabilities enabled us to go much further, towards a better understanding of GRB emission mechanisms and of the GRB phenomenon overall.
In the Fermi Era, the empirical Band function, traditionaly-used to fit GRB spectra, is not the paradigm any more. Deviation from Band and possible identification of physical components are discussed in the litterature.
Through this presentation we will come back on the Fermi observations published in the literature and discuss how they support or challenge the various mechanisms proposed to explain the observed emission.

\index{\tiny{Guiriec, Sylvain: \textit{The Fermi Era: Towards a better understanding of the GRB prompt emission}}}
