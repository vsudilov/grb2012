% $ABSTRACT_SESSION_VII_7


\atitle{GRBs at Neutrino Telescopes}

\bigskip

\authors{Markus Ahlers}

\affiliation{UW Madison}

\bigskip

\noindent Gamma ray bursts are among the prime suspects as the sources of ultra-high energy (UHE) cosmic rays (CRs). Not only are these objects capable of accelerating nuclei to the extreme energies so far detected but they can also supply enough power to sustain the energy density of UHE CRs. The acceleration of nuclei in the source in the presence of radiation fields from the burst or afterglow leads immediately to the production of high energy neutrinos. I will review the expected neutrino fluxes in the standard GRB fireball model. I will show that the IceCube neutrino observatory has now reached the sensitivity to test these fluxes and challenges the UHE CR paradigm of GRBs.
