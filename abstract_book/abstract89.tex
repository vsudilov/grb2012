% $POSTERS_SESSION_PII_p2
P-II-30

\atitle{Evidence of Deterministic Components in the Apparent Randomness of GRBs: Clues of a Chaotic Dynamic}

\bigskip

\authors{Giuseppe Greco}

\affiliation{INAF-Bologna Astronomical Observatory}

\bigskip

\noindent Prompt $\gamma$-ray emissions from gamma-ray bursts (GRBs) exhibit a vast range of extremely complex temporal structures with a typical variability time-scale significantly short – as fast as milliseconds. Despite their morphological complexity, we detect evidence of a non stochastic  variability during the overall burst duration – seemingly consistent with a chaotic behavior. The phase space portrait of such variability shows the existence of a well-defined strange attractor underlying the erratic prompt emission structures. This scenario can shed new light on the ultra-relativistic processes believed to take place in GRB explosions and usually associated with the birth of a fast-spinning magnetar or accretion of matter onto a newly formed black hole.

\index{\tiny{Greco, Giuseppe: \textit{Evidence of Deterministic Components in the Apparent Randomness of GRBs: Clues of a Chaotic Dynamic}}}
