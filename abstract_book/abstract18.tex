% $ABSTRACT_SESSION_IId_2d


\atitle{GRB observations at very high energies with the MAGIC telescopes}

\bigskip

\authors{Markus Garczarczyk}

\affiliation{Instituto Astrofisica de Canarias}

\bigskip

\noindent Measurement of the GRB spectra and variability at the GeV range will constrain the ongoing physical processes of these phenomena. More than a decade ago, when the MAGIC telescope was in the design phase, the main goals were to build an IACT with the fastest repositioning speed and lowest energy threshold. These requirements are mandatory for measuring the VHE emission from GRBs. Currently MAGIC is a stereoscopic system of two telescopes operating at an energy threshold of few tens of GeV. The telescopes move to the opposite direction on the sky within less than 20 s. Follow-up observations are carried out with a frequency of ~10 events per year, however, without a significant detection until now. The MAGIC/LAT results and the modeling of the physical parameters of GRB090102 will be shown.
