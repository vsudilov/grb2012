% $POSTERS_SESSION_PII_p2
P-II-4

\atitle{A Study of Gamma-Ray Bursts with extended emission observed with BATSE}

\bigskip

\authors{Z. Funda Bostanci, Yuki Kaneko, Ersin Gogus}

\affiliation{Faculty of Engineering and Natural Sciences, Sabancı University}

\bigskip

\noindent GRBs are divided into long soft and short hard classes, based on their duration and spectral hardness. Occasionally some GRBs exhibit a softer, low intensity extended emission (EE) component following an initial short-hard spike. These events are considered as a separate population between two classes, although the nature of EE is still unclear. The possibilities for EE include an early X-ray afterglow or a manifestation of the prolong activity of a central engine. During its mission, BATSE recorded 2704 GRBs. Taking advantage of the large BATSE archive we performed a detailed systematic search for short GRBs with EE component. Here we present first results of our search. We study spectral and temporal properties of identified bursts and discuss their nature as a separate class of GRBs.
