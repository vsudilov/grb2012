% $POSTERS_SESSION_PV_p5
P-V-3

\atitle{LGRBs beaming features and SNIb/c connection in the light of the  Second Swift BAT Catalog}

\bigskip

\authors{Chadia Kanaan}

\affiliation{University of Nice-Sophia Antipolis & Observatoire de la Cote d’Azur & Laboratoire Lagrange, UMR 7293, BP 4229, F-06304, Nice Cedex 4, France}

\bigskip

\noindent Assuming a lognormally distributed emitted energy of LGRBs and following a Monte Carlo procedure, we simulate observed fluence distribution and compare it with fluence data selected from the second Swif-BAT Catalog. An evolution of the burst mean energy is required in order to reproduce adequately the redshift distribution.
For our preferred jet emission model is characterized by a linear evolution of the mean energy with redshift and from the resulting energy distribution of detected (simulated) events, a mean energy of log EJ =49.48 (in erg) is derived.The estimated local formation rate is Rgrb =290 Gpc−3 yr−1 representing less than 9% of the local formation rate of type Ibc supernovae.
