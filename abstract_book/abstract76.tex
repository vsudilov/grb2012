% $POSTERS_SESSION_PII_p2
P-II-19

\atitle{GRB 080407: an ultra-long burst discovered by the IPN}

\bigskip

\authors{V. Pal'shin [1], K. Hurley [2], J. Goldsten [3], I. G. Mitrofanov [4], W. Boynton [5], A. von Kienlin [6], J. Cummings [7], M. Feroci [8],   R. Aptekar [1], D. Frederiks [1], S. Golenetskii [1], E. Mazets [1], D. Svinkin [1], D. Golovin [4], M. L. Litvak [4], A. B. Sanin [4], C.   Fellows [5], K. Harshman [5], R. Starr [5], A. Rau [6], V. Savchenko [6], X. Zhang [6], S. Barthelmy [7], N. Gehrels [7], H. Krimm [7], D.  Palmer [7], E. Del Monte [8], M. Marisaldi [8]}

\affiliation{[1] for the Konus-Wind team, [2] for the IPN, [3] for the MESSENGER GRNS team, [4] for the HEND Mars Odyssey team, [5] for the GRS Mars Odyssey team, [6] for the INTEGRAL SPI-ACS team, [7] for the Swift BAT team, [8] for the SuperAGILE and AGILE MCAL teams}

\bigskip

\noindent We present observations of the extremely long GRB 080704 obtained with the instruments of the Interplanetary Network (IPN). The observations reveal two distinct emission episodes, separated by a ~1500 s long period of quiescence. The total burst duration is about 2100 s. We compare the temporal and spectral characteristics of this burst with those obtained for other ultra-long GRBs and discuss these characteristics in the context of different models.
