% $ABSTRACT_SESSION_IIc_2c


\atitle{On The Lack of Time Dilation Signatures in Gamma-ray Burst Light Curves}

\bigskip

\authors{Daniel Kocevski}

\affiliation{Stanford University}

\bigskip

\noindent We present an analysis in which we examine the effects of time dilation on the temporal profiles of GRB pulses and the implications this has on our understanding of GRB energetics.  Through the use of simulations, we find that the observer frame duration of individual pulses does not increase with redshift as one would expect from cosmological expansion. Instead, the duration actually decreases as the pulse's signal-to-noise decreases with increasing redshift.  The results of our simulation are consistent with the fact that evidence for time dilation has not materialized in either the Swift or Fermi detected GRBs with known redshift.  We discuss how the measured durations and E$_{\rm iso}$ estimates for GRBs detected near the instrument's detection threshold should be considered as lower limits.

\index{\tiny{Kocevski, Daniel: \textit{On The Lack of Time Dilation Signatures in Gamma-ray Burst Light Curves}}}
