% $POSTERS_SESSION_PVI_p6
P-VI-1

\atitle{The HAWC experiment and its sensitivity to gamma-ray bursts}

\bigskip

\authors{D. Zaborov, for the HAWC collaboration}

\affiliation{Pennsylvania State University, USA}

\bigskip

\noindent The High Altitude Water Cherenkov Observatory (HAWC) is an air shower array currently under construction in Mexico at an altitude of 4100 m. HAWC will consist of 300 large water tanks covering an area of about 22000 square meters and instrumented with 4 photomultipliers each. The experimental design allows for highly efficient detection of photon-induced air showers in the TeV and sub-TeV range and gamma-hadron separation. We show that HAWC has a realistic opportunity to observe the high-energy power law components of GRBs that extend at least up to 30 GeV. In particular, HAWC will be capable to observe events similar to GRB 090510 and GRB 090902b. The observations (or non-observations) of GRBs by HAWC will provide information on the high-energy spectra of GRBs.
