% $POSTERS_SESSION_PII_p2
P-II-20

\atitle{Where is the photospheric emission in GRBs ?}

\bigskip

\authors{R. Hascoet, F. Daigne and R. Mochkovitch}

\affiliation{Institut d'Astrophysique de Paris}

\bigskip

\noindent The lack of bright black body components in most observed GRB spectra seems in contradiction with the standard fireball model, where the initial energy content of the flow is essentially thermal. Here we investigate an alternative scenario where the GRB flow is initially magnetically dominated but contains a subdominant thermal component that cools passively up to the photosphere. We compute the resulting thermal emission and we consistently superimpose a non-thermal contribution produced either by internal shocks or by magnetic dissipation, depending on the remaining magnetization of the jet after its acceleration. Within this picture, we explore the parameter space and compare predictions (light-curves and spectra) with different tentative detections of thermal components in GRB spectra.

\index{\tiny{Hascoet, Romain: \textit{Where is the photospheric emission in GRBs ?}}}
