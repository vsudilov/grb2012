% $ABSTRACT_SESSION_VI_6


\atitle{The History and Legacy of BATSE}

\bigskip

\authors{Gerald J. (Jerry) Fishman}

\affiliation{NASA-Marshall Space Flight Center}

\bigskip

\noindent The BATSE experiment on the Compton Gamma-ray Observatory was the first large detector system specifically designed for the study of gamma-ray bursts. The eight large-area detectors allowed full-sky coverage and were optimized to operate in the energy region of the peak emission of most GRBs.  BATSE provided detailed observations of the temporal and spectral characteristics of large samples of GRBs, and it was the first experiment to provide rapid notifications of the coarse location of many them.  It also provided strong evidence for the cosmological distances to GRBs through the observation of the sky distribution and intensity distribution of numerous GRBs.  The origin and development of the BATSE experiment, some highlights from the mission and its continuing legacy will be described.
