% $POSTERS_SESSION_PVI_p6
P-VI-4

\atitle{The investigation of Gamma Ray Bursts with the Large Observatory  for x-ray Timing (LOFT)}

\bigskip

\authors{E. Del Monte [1], L. Amati [2] on behalf of the LOFT team}

\affiliation{[1] INAF IAPS Roma, [2] INAF IASF Bologna}

\bigskip

\noindent LOFT is a satellite mission currently in Assessment Phase for the
ESA M3 selection. The payload is composed of the Large Area
Detector (LAD), with 2 - 50 keV energy band, a peak effective area
of about 10 m^2 and the energy resolution better than 260 eV, and
the Wide Field Monitor (WFM), a coded mask imager with an energy
resolution of about 300 eV and a point source location accuracy of
1 arcmin in the 2 - 50 keV energy range.

The scientific performances of the WFM are particularly suited to
investigate the most relevant open issues in the study of GRBs: the
physics of the prompt emission, the spectral absorption features by
circum-burst material (and hence the nature of the progenitors),
the population and properties of XRFs, and the detection and rate of high-z GRBs.
