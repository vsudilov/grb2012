% $ABSTRACT_SESSION_IIc_2c


\atitle{The Ep - Eiso relation: intrinsic GRB property or/and selection effects?}

\bigskip

\authors{R. Mochkovitch, R. Hascoet, F. Daigne, J.L. Atteia, S. Boci, M. Hafizi}

\affiliation{Institut d'Astrophysique de Paris}

\bigskip

\noindent Since its discovery the Ep - Eiso relation has been a source of controversy between those believing that it is an intrinsic GRB property, that should be directly related to the physics of the prompt emission and others, arguing that it comes from various selection effects (thresholds for burst and afterglow detection, conditions for measuring the redshift and peak energy). In the context of the internal shock model we present Monte-Carlo simulations showing that the observed relation may indeed result from a combination of intrinsic processes and selection effects. While at large Eiso we find that the relation can be interpreted in terms of the physics of internal shocks (which do not produce large Eiso - low Ep events), it appears largely shaped by selection effects at low Eiso.
