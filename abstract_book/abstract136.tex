% $POSTERS_SESSION_PVII_p7
P-VII-3

\atitle{Gamma Ray Burst Cosmology with Gravitational Waves}

\bigskip

\authors{Xihao Deng}

\affiliation{The Pennsylvania State University}

\bigskip

\noindent Gamma ray bursts (GRBs) are the brightest sources in the universe. With the discovery of correlations between their peak energy and isotropically equivalent gamma ray luminosity or energy, they are being considered as 'standard candle' to measure cosmological parameters. However, this method usually cannot give a precise result and it might have a 'circular' problem. With the recent discovery of the correlation between time resolved peak energy and isotropically equivalent gamma ray luminosity, which is applicable to both long and short bursts, we find that detecting gravitational waves (GWs) from short gamma ray bursts by LIGO can help calibrate the GRB sample and precisely determine the cosmological parameters. Gravitational wave detection from short GRBs can directly determine their luminosity distances and correspondingly their time resolved luminosity without referring to their redshifts. Therefore, we can take advantage of the GW method to calibrate short GRBs with distant long GRBs by their time resolved peak energy and gamma ray luminosity correlation, and thus extend the GRB 'standard candles' to high redshift. This technique will help construct a precise high redshift Hubble diagram and will be complementary with the Type Ia Supernovae counterpart.

\index{\tiny{Deng, Xihao: \textit{Gamma Ray Burst Cosmology with Gravitational Waves}}}
