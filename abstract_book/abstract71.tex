% $POSTERS_SESSION_PII_p2
P-II-14

\atitle{GRBs in the comoving frame: the link between the jet opening angle and the bulk Lorentz factor and the interpretation of the spectral-energy correlations.}

\bigskip

\authors{G. Ghirlanda}

\affiliation{INAF-Osservatorio Astronomico di Brera}

\bigskip

\noindent I will present the comoving frame properties of GRBs whose $\Gamma_0$ can be estimated from the peak of their afterglow light curve. Typical values of $\Gamma_0$ are $\sim$138 and $\sim$66, if the circum-burst density is constant or if it scales as a wind profile respectively. The distribution of the comoving frame Liso is very narrow  with a typical value $\sim$5e48 erg/s. Intriguingly, there exist tight correlations between E$_{\rm iso}$ or L$_{\rm iso}$ and $\Gamma_0$ which offer a general interpretation scheme for the spectral--energy correlation of GRBs: the E$_p$-E$_{\rm iso}$ and E$_p$-L$_{\rm iso}$ correlations are due to a sequence of $\Gamma_0$ factors. The collimation-corrected correlation, can be explained  if $\theta^2 \times \Gamma_0$=constant. Finally, we find a typical $\theta \times \Gamma_0$ $\sim$ 6-20 in agreement with magnetically accelerated jets models.

\index{\tiny{Ghirlanda, Giancarlo: \textit{GRBs in the comoving frame: the link between the jet opening angle and the bulk Lorentz factor and the interpretation of the spectral-energy correlations.}}}
