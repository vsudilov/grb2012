% $POSTERS_SESSION_PII_p2
P-II-13

\atitle{Average Power Density Spectrum of long GRBs in the Swift Era}

\bigskip

\authors{C. Guidorzi [1], R. Margutti [2], L. Amati [3], S. Campana [4], M. Orlandini [3], P. Romano [5], M. Stamatikos [6], G. Tagliaferri [4]}

\affiliation{[1] University of Ferrara, [2] Harvard Univ, [3] INAF-IASF(Bologna), [4] INAF-OAB, [5] INAF-IASF(Palermo), [6] NASA-GSFC}

\bigskip

\noindent From past experiments the average power density spectrum of GRBs with unknown redshift was found to be modelled from 0.01 to 1 Hz with a power-law, f$^{-\alpha}$, with $\alpha$ $\sim$5/3. In hydrodynamics this is suggestive of a Kolmogorov spectrum of velocity fluctuations within a medium with
fully-developed turbulence. We updated the same analysis to the sample of Swift GRBs, and for the first time we calculated the average PDS in the GRB source rest frame, correcting for the time and energy dependence of GRB time histories.
For the first time it was possible to study the average PDS as a function of different intrinsic properties and redshift.
We present the main results and implications of this investigation.

\index{\tiny{Dichiara, Simone: \textit{Average Power Density Spectrum of long GRBs in the Swift Era}}}
