% $POSTERS_SESSION_PII_p2
P-II-26

\atitle{Evidence for the observation of the first genuine short GRB and theoretical inference of its cosmological redshift.}

\bigskip

\authors{Marco Muccino [1], Remo Ruffini [1,2], Carlo Luciano Bianco [2], Luca Izzo [1,2], Ana Virginia Penacchioni [1]}

\affiliation{[1] ICRANet and Sapienza University of Rome, [2] ICRANet - International Center for Relativistic Astrophysics Network}

\bigskip

\noindent The recent observations by the Fermi GBM detectors have given the possibility to study for
the first time a genuine short GRB, GRB 090227B, with energy E=(1.13+/-0.12)*10^54 ergs and
baryon load B=10^{-6}. Under such conditions, the theoretical understanding of the electron-
positron plasma allows to estimate the Lorentz gamma factor at transparency, Gamma=12852, and, consequently, to determine its cosmological redshift, z=4.07+/-0.36.

\index{\tiny{Muccino, Marco: \textit{Evidence for the observation of the first genuine short GRB and theoretical inference of its cosmological redshift.}}}
