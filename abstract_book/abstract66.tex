% $POSTERS_SESSION_PII_p2
P-II-1

\atitle{Variation of the Rest-Frame Lag-Luminosity Relation with Redshift for Long Swift GRBs}

\bigskip

\authors{Walid J. Azzam [1], Hala A. Eid [2]}

\affiliation{[1] University of Bahrain, [2] Bahrain Polytechnic}

\bigskip

\noindent A sample of long Swift gamma-ray bursts (GRBs) is used to investigate the possible dependence of the lag-luminosity relation on redshift in the burst's rest frame. The lag-luminosity relation is basically an anti-correlation between the time lag, T, which represents the delay between the arrival of hard and soft photons, and the isotropic peak luminosity, Lp. Our method consists of binning the data in redshift, z, then applying a fit of the form: log(L$_{\rm p}$) =A + B log(T$_{\rm 0}$/T$_{\rm mean}$) for each bin, where T$_{\rm 0}$ is the time-lag in the burst’s rest frame, and T$_{\rm mean}$ is the corresponding mean value for the entire sample. The objective is to see whether the two fitting parameters, A and B, depend on redshift. Our results indicate that the normalization, A, and the slope, B, do show a redshift dependence.

\index{\tiny{Azzam, Walid: \textit{Variation of the Rest-Frame Lag-Luminosity Relation with Redshift for Long Swift GRBs}}}
