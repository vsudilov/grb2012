% $POSTERS_SESSION_PV_p5
P-V-2

\atitle{Chemical properties of long gamma-ray bursts progenitors in cosmological simulations}

\bigskip

\authors{L. J. Pellizza [1,2], M. C. Artale [1,2] & P. B. Tissera [1,2]}

\affiliation{[1] Instituto de Astronomía y Física del Espacio, Ciudad de Buenos Aires, Argentina, [2] Consejo Nacional de Investigaciones Científicas y Técnicas, Argentina}

\bigskip

\noindent We investigate the chemical dependence of the progenitors of long gamma-ray bursts (LGRBs). Using hydrodynamical cosmological simulations consistent with the concordance $\\Lambda$-CDM model which include star formation, chemical enrichment and Supernovae feedback in a self-consistent way, and assuming that LGRBs are produced by massive stars with a possible chemical dependence, we compute the LGRB rate at different redshifts. Introducing a prescription for their peak luminosity function and intrinsic spectrum, and using a Monte Carlo scheme to model their detectability by different high-energy observatories, we compute the distribution of their observables (peak flux, spectral peak energy). In this poster we present our preliminary results compared with current observations.
