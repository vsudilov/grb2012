% $POSTERS_SESSION_PIII_p3
P-III-13

\atitle{Energetic Fermi/LAT GRB100414A: Energetic and Correlations}

\bigskip

\authors{Yuji Urata [1], Kuiyun Huang [2], Kazutaka Yamaoka [3],Patrick P. Tsai [1], Makoto S. Tashiro [4]}

\affiliation{[1] NCU, [2] ASIAA, [3] Aoyama Gakuin Univ., [4] Saitama Univ.}

\bigskip

\noindent This study presents multi-wavelength observational results for energetic
GRB100414A with GeV photons. The prompt spectral fitting using Suzaku/WAM data
yielded spectral peak energies of E$_{\rm peak}$ of 1458.7 (+132.6, -106.6) keV and
E$_{\rm iso}$ of 34.5(+2.0, -1.8) $\times$ 10$^{52}$ erg with z=1.368. The optical afterglow light
curves between 3 and 7 days were effectively fitted according to a simple power
law with a temporal index of $\alpha=-2.6 \pm 0.1$. The joint light curve with
earlier Swift/UVOT observations yields a temporal break at 2.3 $\pm$ 0.2 days.
This was the first Fermi/LAT detected event that demonstrated the clear
temporal break in the optical afterglow. The jet opening angle derived from
this temporal break was 5.8 degree, consistent with those of other
well-observed long gamma-ray bursts (GRB

\index{\tiny{Kuiyun, Huang: \textit{Energetic Fermi/LAT GRB100414A: Energetic and Correlations}}}
