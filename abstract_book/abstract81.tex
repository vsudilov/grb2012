% $POSTERS_SESSION_PII_p2
P-II-23

\atitle{Canonical GRBs: the long, the disguised short and the short, and their cosmic distances.}

\bigskip

\authors{R. Ruffini [1,2,6], M.G. Bernardini [3,2], C.L. Bianco [1,2], L. Caito [1,2], G. De Barros [1,2], L. Izzo [1,2], M. Muccino [1,2], B. Patricelli [5,1,2], A.V. Penacchioni [4,1,2], G.B. Pisani [4,1,2], I. Siutsou [2], G. Vereshchagin [2]}

\affiliation{[1] Dipartimento di Fisica and ICRA, Universita' di Roma "La Sapienza", [2] ICRANet, [3] Italian National Institute for Astrophysics (INAF) - Osservatorio Astronomico di Brera, [4] Erasmus Mundus Joint Doctorate IRAP PhD. Student, [5] Astronomy Institute - UNAM, [6] ICRANet, Universite' de Nice Sophia Antipolis }

\bigskip

\noindent Following each major mission, the understanding of GRBs has conceptually evolved. From the definition of long and short GRBs after BATSE, to the determination of the distance and energetics following BeppoSAX, pointing to isotropic energies in the range between 10^{49} and 10^{55} ergs and to black holes as their energy sources, the observations by Swift have given the first evidence of a new class of \"disguised\" short GRBs. These sources, with their sharp spike followed by a prolongued soft tail, appear to be canonical long GRBs exploding in a very low density environment (10^{-3} particles/cm^3). They therefore point to binary mergers in galactic halos as their progenitors. The Fermi satellite as well as the indian-russian satellite Coronas-Photon have given the first evidence for the single core collapse of a proto-black hole followed by a canonical long GRB emission. The theoretical understanding of this phenomena is allowing to use these source as standard candles. Members of this family at selected redshifts z = 0.54 for GRB 090618, at z ~ 0.9 estimated for GRB 101023 all the way up to z ~ 8 for GRB 090423 are currently studied. It is clear, therefore, that GRBs are not all different: at least two very different families exist, the one related to merging binary sources and the one corresponding to a single core collapse. The cosmological distribution of such sources will be discussed, as well as the evidence for the observation of the first genuine short GRB.

\index{\tiny{Ruffini, Remo: \textit{Canonical GRBs: the long, the disguised short and the short, and their cosmic distances.}}}
