% $ABSTRACT_SESSION_IIc_2c


\atitle{On the reliability of spectral-energy correlations in GRBs}

\bigskip

\authors{L. Amati [1], S. Dichiara [2], F. Frontera [2], C. Guidorzi [2]}

\affiliation{[1] INAF - IASF Bologna, [2] University of Ferrara}

\bigskip

\noindent The E$_{\rm p,i}$ - E$_{\rm iso}$ correlation, first reported by Amati et al. (2002), and other spectral-energy correlations derived from it are one of the most debated topics in GRB science, given their relevance for both GRB physics and cosmology. In particular, in the last years different research groups came to very different conclusions about the impact of selection and instrumental effects on these correlations. We give a critical review of these works and report the results of Monte Carlo simulations and the analysis of large datasets (CGRO/BATSE, BeppoSAX/GRBM, WIND/Konus, Swift/BAT, Fermi/GBM) that we performed in order to shade light on this important issue. We also discuss the reliability and perspectives of using spectral-energy correlations for measuring cosmological parameters up to z$\sim$9.

\index{\tiny{Amati, Lorenzo: \textit{On the reliability of spectral-energy correlations in GRBs}}}
