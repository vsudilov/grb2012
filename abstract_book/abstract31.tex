% $ABSTRACT_SESSION_IVa_4a


\atitle{High-redshift Gamma-ray Burst Afterglows}

\bigskip

\authors{N. Tanvir}

\affiliation{University of Leicester}

\bigskip

\noindent The brightest GRB afterglows could, in principle, be seen at very high redshifts, during and before the epoch of reionization.  I will discuss the promise of GRB afterglows for pinpointing the locations of their faint host galaxies, and providing, via spectroscopy, information about their environments and the IGM that cannot be obtained by other means.  I will also review progress to-date in locating and studying GRB afterglows at very high-z, and consider prospects for the future.
