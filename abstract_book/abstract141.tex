% $POSTERS_SESSION_PVIII_p8
P-VIII-3

\atitle{The Metallicity Aversion of Long Duration Gamma-Ray Bursts}

\bigskip

\authors{Andrew Fruchter, John Graham}

\affiliation{STScI and Johns Hopkins University}

\bigskip

\noindent It has been suggested that the apparent bias of long-duration GRBs (LGRBs) to low metallicity environments might be a result of the fact that star-formation is anti-correlated with metallicity.   However, if this were the cause, one would expect other indicators of star formation, such as Type II and Type Ic SNe to demonstrate a similar bias.  Here we show that local Type Ic and Type II SNe track the star-formation weighted metallicity distribution of the SDSS galaxies.  In contrast LGRBs are typically found at far-lower metallicities than would be expected based on the distribution of star-formation.   This is true even when one takes into account so-called \\\"dark bursts\\\".  The bias of LGRBs to significantly sub-solar metallicity must be related to a mechanism crucial to their formation.

\index{\tiny{Fruchter, Andrew: \textit{The Metallicity Aversion of Long Duration Gamma-Ray Bursts}}}
