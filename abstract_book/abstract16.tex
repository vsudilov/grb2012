% $ABSTRACT_SESSION_IId_2d


\atitle{Study of emission mechanism of Gamma-Ray Bursts by the gamma-ray polarization with IKAROS-GAP}

\bigskip

\authors{Daisuke Yonetoku [1], Toshio Murakami [2], Shuichi Gunji [2], Tatehiro Mihara [3], Kenji Toma [4], Tomonori Sakashita [1], Yoshiyuki Morihara [1], Takuya Takahashi [1], Noriyuki Toukairin [2], Hirofumi Fujimoto [1], Yoshiki Kodama [1]}

\affiliation{[1] Kanazawa University, [2] Yamagata University, [3] RIKEN, [4] Osaka University}

\bigskip

\noindent The Gamma-Ray Burst Polarimeter (GAP) aboard the solar sail IKAROS is the first polarimeter specifically designed to measure the gamma-ray polarization of prompt GRBs. We detected the polarization signals from three bright GRBs with almost above 3 sigma confidence level. For the case of GRB100826A, we also detected the firm change of polarization angle with 3.5 sigma confidence level. In view of gamma-ray polarization measurement, we consider that the magnetic fields may play an important role in the mechanism of prompt GRB. We suggest the non-axisymmetric (e.g. patchy) structures of the magnetic fields and/or brightness inside the relativistic jet within the observable angular scale of $\\Gamma^{-1}$, to explain the observed significant change of polarization angle.
