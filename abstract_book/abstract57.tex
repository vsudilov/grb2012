% $ABSTRACT_SESSION_VI_6


\atitle{Hunting for Gamma Ray Bursts with Pi of the Sky telescopes in Chile and Spain}

\bigskip

\authors{M.Siudek et al. (Pi of the Sky Collaboration)}

\affiliation{Center for Theoretical Physics PAS}

\bigskip

\noindent Pi of the Sky is a system of  wide field-of-view  robotic telescopes, which search for short timescale astrophysical phenomena, especially for prompt optical GRB emission. The system was designed for autonomous operation, following the predefined observation strategy and adopting it to the actual conditions. Simultaneous observations from locations in Chile and Spain allows for a systematic search of optical transients of cosmological origin.
Accurate analysis of data arising from a wide-field system like Pi of the Sky  is a real challenge because of a number of factors that can influence measurements. We have developed a set of dedicated algorithms which remove poor quality measurements, improve photometric accuracy and allow us to reach uncertainties  as low as 0.015 – 0.02 mag.

\index{\tiny{Siudek, Malgorzata: \textit{Hunting for Gamma Ray Bursts with Pi of the Sky telescopes in Chile and Spain}}}
