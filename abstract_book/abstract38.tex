% $ABSTRACT_SESSION_IVb_4b


\atitle{Probing Cosmic Radiation Fields and Magnetic Fields in the Reionization Epoch with Hard X-rays and MeV-GeV Gamma Rays from High-z GRBs}

\bigskip

\authors{Susumu Inoue [1], Yoshiyuki Inoue [2], Masakazu A. R. Kobayashi [3], Tomonori Totani [2], Keitaro Takahashi [4], Kiyotomo Ichiki [5]}

\affiliation{[1] ICRR, U. Tokyo, [2] Kyoto U., [3] NAOJ, [4] Kumamoto U., [5] Nagoya U.}

\bigskip

\noindent The extragalactic background light (EBL) in the rest-frame UV plays a crucial role during cosmic reionization. Based on semi-analytic models of galaxy formation, we develop a model of the EBL for z=0-10 that include Pop III stars and are consistent with current reionization constraints from QSO spectra and CMB polarization. The EBL can be probed via attenuation features due to gamma-gamma interactions in the 10-100 GeV spectra of high-z GRBs. Furthermore, the electron-positron pairs produced in such interactions can induce delayed secondary emission by scattering the CMB up to hard X-ray to multi-MeV energies, which provide a unique probe of intergalactic magnetic fields as well as the EBL during reionization. We discuss the observational prospects for CTA, ASTRO-H and other instruments.

\index{\tiny{Inoue, Susumu: \textit{Probing Cosmic Radiation Fields and Magnetic Fields in the Reionization Epoch with Hard X-rays and MeV-GeV Gamma Rays from High-z GRBs}}}
