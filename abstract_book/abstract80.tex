% $POSTERS_SESSION_PII_p2
P-II-22

\atitle{Needs for a new GRB classification following the fireshell model: "genuine short", "disguised short" and "long" GRBs}

\bigskip

\authors{C.L. Bianco [1,2], M.G. Bernardini [3,2], L. Caito [1,2], G. De Barros [1,2], L. Izzo [1,2], M. Muccino [1,2], B. Patricelli [5,1,2], A.V. Penacchioni [4,1,2], G.B. Pisani [4,1,2], R. Ruffini [1,2,6]}

\affiliation{[1] Dipartimento di Fisica and ICRA, Universita' di Roma "La Sapienza", [2] ICRANet, [3] Italian National Institute for Astrophysics (INAF) - Osservatorio Astronomico di Brera, [4] Erasmus Mundus Joint Doctorate IRAP PhD. Student, [5] Astronomy Institute - UNAM, [6] ICRANet, Universite\' de Nice Sophia Antipolis}

\bigskip

\noindent Thanks to the observations by Swift and Fermi, it became clear that the traditional classification of GRBs in two classes ("short/hard" and "long/soft") is, at best, misleading. Following the theoretical approach of the fireshell model, we discuss the needs for a new GRB classification based on intrinsic physical properties which includes at least three classes: \"genuine short\", \"disguised short\" and \"long\", whose names resembles the traditional ones just for simplicity. Observational facts supporting this classification will be discussed, including the implications for the \"E_{p,i}-E_{iso}\" correlation.
