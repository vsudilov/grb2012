% $POSTERS_SESSION_PV_p5
P-V-4

\atitle{Late-time light curves of GRB associated supernovae}

\bigskip

\authors{Kuntal Misra, Andrew S. Fruchter}

\affiliation{Space Telescope Science Institute}

\bigskip

\noindent We examine two GRB/XRF related SNe, 2003dh and 2006aj, which are unique because of the availability of the HST data allowing us to constrain the decay nature of the late-time light curves.The multi-color HST observations extend up to ~400 days after the burst. The decay rates seen from the HST data are steeper than the $^{56}Co->^{56}Fe$ decay rates indicating that there is some leakage of Υ-rays. We compare the late-time light curves of type Ic SNe population including the SNe with/without an observed GRB and normal type Ic. The late-time light curves display a wide range of decay rates adding to the diversity of type Ic SNe. The scaled luminosity, with respect to the unit mass of ejected 56Ni, of type Ic SNe forms a tight group in the I band whereas the B band displays a large scatter.

\index{\tiny{Misra, Kuntal: \textit{Late-time light curves of GRB associated supernovae}}}
