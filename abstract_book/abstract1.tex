% $ABSTRACT_SESSION_I_1


\atitle{Fermi LAT Status and Recent Results}

\bigskip

\authors{F. Piron, for the Fermi LAT Collaboration}

\affiliation{LUPM (CNRS/IN2P3 and Montpellier 2 University)}

\bigskip

\noindent The Fermi Large Area Telescope (LAT) is a pair-conversion detector of high-energy gamma rays of energies ranging from 20 MeV to more than 300 GeV. It operates in synergy with the Gamma-ray Burst Monitor (GBM), which covers the entire unocculted sky and is designed for gamma-ray transients' detection and spectroscopy between 8 keV and 40 MeV. The LAT detects $\sim$9 gamma-ray bursts (GRB) per year, which represents $\sim$8\% of the GBM bursts occuring in its field of view. We will present a systematic study of the GRB properties observed with the LAT, highlighting the unique characteristics of some individual bursts. We will also discuss the LAT non-detections of some GBM bright and hard bursts, in an effort to understand the difference in GRB rates from both instruments.

\index{\tiny{Piron, Frederic: \textit{Fermi LAT Status and Recent Results}}}
