% $POSTERS_SESSION_PVIII_p8
P-VIII-2

\atitle{GRB 100206A:  The first short GRB associated with recent star-formation?}

\bigskip

\authors{Daniel Perley}

\affiliation{Caltech}

\bigskip

\noindent The known host galaxies of short-hard gamma-ray bursts (GRBs) to date are characterized by low star-formation rates and a broad range of stellar masses.  In this work we positionally associate short-hard GRB 100206A with a luminous infrared galaxy (LIRG) whose star-formation rate is $\sim30 _{\odot} $yr$^{-1}$, almost an order of magnitude higher than any previously identified short GRB host.  While these properties could be interpreted to support an association of this GRB with very recent star formation, the broadband SED shows that a substantial mass of older stars is present, resulting in a relatively modest specific SFR (0.5 Gyr$^{-1}$). Our observations are therefore equally consistent with an older progenitor, similar to what is  inferred for other short-hard GRBs.

\index{\tiny{Perley, Daniel: \textit{GRB 100206A:  The first short GRB associated with recent star-formation?}}}
