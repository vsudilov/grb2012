% $ABSTRACT_SESSION_VI_6


\atitle{The Initial Gamma Ray Transient Studies}

\bigskip

\authors{Thomas L. Cline}

\affiliation{Goddard Space Flight Center, Emeritus}

\bigskip

\noindent The early studies of astronomical gamma ray transients are reviewed, with emphasis on our involvements and related developments, e.g., the first gamma ray burst detections and the creation of the Interplanetary Network. I interpreted the 1979 March 5 event as unique, whereas this initial IPN success appeared to ironically complicate and deepen the mystery of GRBs, with other views that its N49 snr id. was accidental or that it suggested nearby n-star origins. Other early experiments are outlined, e.g. our Shuttle Gamma-Dome that inspired the CGRO-BATSE, the Ge GRB spectrometer, and the Solar Polar Mission, reduced to the successful but no longer stereoscopic Ulysses. Other developments prior to the cosmological GRB era, e.g. the resolution of the magnetar/GRB confusion, are outlined.
