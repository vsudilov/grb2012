% $POSTERS_SESSION_PII_p2
P-II-31

\atitle{Background fitting of Fermi gamma-ray burst 091030613}

\bigskip

\authors{Dorottya Szécsi [1], Zsolt Bagoly [1,2], István Horváth [2], Lajos G. Balázs [3], Péter Veres [1,2,3], Attila Mészáros [4]}

\affiliation{[1] Eötvös University, Budapest, [2] Bolyai Military University, Budapest, [3] Konkoly Observatory, Budapest, [4] Charles University, Prague}

\bigskip

\noindent Fermi Gamma-ray Burst Monitor (GBM) detects gamma-rays between the
energy range 8 keV - 40 MeV. Background fitting of the Fermi data is
not trivial in some cases, especially when an Autonomous Repoint
Request (ARR) is received. One good exemple is the burst 091030613
measured by the GBM, which cannot be well fitted by a third-order
polynom of time. We present the background fitting of this burst for
energy channels given in the CTIME data-file. Our method is based on
the proper motion of the satellite: we define 3 underlying parameters
which depend on the actual position and orientation of the satellite
and use them to fit the background. Main steps and results of this
process are shown on the poster for the triggered NaI 3 detector.
