% $POSTERS_SESSION_PII_p2
P-II-28

\atitle{Curvature Effects in GRBs}

\bigskip

\authors{A. Shenoy [1], E. Sonbas [2,3], C. Dermer [5], L. C. Maximon [1], J. Hakkila [6], P. N. Bhat [7], K.S. Dhuga [1], T. N. Ukwatta [1,2,4], G. A. MacLachlan [1], W. C. Parke [1]}

\affiliation{[1] Department of Physics, The George Washington University, [2] NASA Goddard Space Flight Center [III] University of Adiyaman, [IV] Department of Physics and Astronomy, Michigan State University, [V] Naval Research Laboratory, [VI] College of Charleston, [VII] University of Alabama, Huntsville}

\bigskip

\noindent Spectral lags in Gamma-ray Bursts (GRBs), wherein low energy photons
arrive at later times than the high energy photons, have been observed
in a significant number of bursts. Spectral lags have been attributed
to curvature effects in GRBs. In this scenario, the relativistic GRB
jet is assumed to be a cone with a certain opening angle. Photons that
are off-axis relative to the observer arrive at later times than
on-axis photons. In this work, we test GRB data for spectral lags and
other such curvature effects by invoking a relatively simple
kinematic, two-shell collision model for a uniform jet profile. We
have isolated single pulse structures in the BATSE, Swift-BAT, and
Fermi-GBM GRB samples.  We present the preliminary results of our
analysis.
