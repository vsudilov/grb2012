% $POSTERS_SESSION_PVII_p7
P-VII-2

\atitle{Neutrinos and Gamma Rays from the First Ultra High Energy Cosmic Rays in the Universe}

\bigskip

\authors{Susumu Inoue [1], Ruo-Yu Liu [2], Xiang-Yu Wang [2], Felix Aharonian [3,4]}

\affiliation{[1] ICRR, U. Tokyo, [2] Nanjing U., [3] MPIK, [4] Dublin IAS}

\bigskip

\noindent Population III stars may give rise to GRBs with much greater energy compared to ordinary ones, and they may also produce ultra-high energy cosmic rays (UHECRs) with a correspondingly larger energy budget. Despite their typically high redshifts, the consequent flux of cosmogenic neutrinos and cascade gamma rays due to interactions of the UHECRs with background radiation fields may be potentially observable. Under different assumptions for the Pop III GRB energetics and formation rate, we show that current upper limits from IceCube-40 and/or the observed gamma-ray background by Fermi-LAT already rule out the most optimistic expectations. Future observations by the full IceCube may possibly detect such neutrinos and offer a unique probe of the very first UHECRs that appear in the Universe.

\index{\tiny{Inoue, Susumu: \textit{Neutrinos and Gamma Rays from the First Ultra High Energy Cosmic Rays in the Universe}}}
