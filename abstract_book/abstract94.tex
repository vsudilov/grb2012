% $POSTERS_SESSION_PII_p2
P-II-5

\atitle{GRB Spectral Lags in The Source Frame: An Investigation of Fermi-GBM Bursts}

\bigskip

\authors{E. Sonbas [1], T.N. Ukwatta [2], K.S. Dhuga [3], A. Shenoy [3], N. Bhat [4], C. Dermer [5], J. Hakkila [6], G. Maclachlan [3] , N. Gehrels [7], L.C. Maximon [3], W. C. Parke [3]}

\affiliation{[1] Adiyaman Univ., [2] Michigan State University, [3] George Washington Univ., [4] NASA-MSFC, [5] Naval Research Laboratory, [6] Univ. of Charleston, [7] NASA-GSFC}

\bigskip

\noindent We have extracted spectral lags of Fermi GBM bursts in eight source-frame energy bands. Our sample contains 37 GRBs with measured redshifts in the range 0.49 - 8.26. The source-frame energy bands were obtained by projecting variable observer-frame bands to the source-frame. The spectral lags were calculated by fitting the global maximum in the Cross Correlation Function (CCF) by a gaussian function. The uncertainties in the spectral lags were estimated using a Monte Carlo Simulation. We compare our results with the existing Swift results (Ukwatta et al. (2011)). We find that the Lag - Luminosity and Epk - Luminosity relations follow a trend similar to that established by the Swift data. 
