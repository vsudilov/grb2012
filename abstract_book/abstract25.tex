% $ABSTRACT_SESSION_IIIb_3b


\atitle{Probing GRB afterglows with deep linear and circular polarimetry}

\bigskip

\authors{K. Wiersema}

\affiliation{University of Leicester}

\bigskip

\noindent Follow-up observations of large numbers of GRBs have produced a large sample of spectra and lightcurves, from which the basic micro- and macrophysical parameters of afterglows may be derived. However, a number of observed phenomena defy explanation by simple versions of the standard fireball model. Polarimetry can be a powerful diagnosis of (new) afterglow physics, probing the magnetic field properties and geometrical effects (e.g. jet breaks). 

We present the first high quality, multicolour, polarimetric curve of a Swift GRB afterglow, using the VLT. We obtained linear polarimetry in R band (0.13 - 2.3 days after burst) as well as K band, and deep circular polarimetry. We combine this unique dataset with multicolour lightcurves to probe models of jet breaks and magnetic field generation.

\index{\tiny{Wiersema, Klaas: \textit{Probing GRB afterglows with deep linear and circular polarimetry}}}
