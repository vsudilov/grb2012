% $POSTERS_SESSION_PIII_p3
P-III-9

\atitle{Identifying the Location in the Host Galaxy from a Short GRB 111117A  by the Chandra Sub-arcsecond Position}

\bigskip

\authors{T. Sakamoto [1,2,3], E. Troja [2,4], K. Aoki [5], P. D’Avanzo [6],  J. Gorosabel [8], S. Guiriec [2,4], K.Y. Huang [10], M. Im [11],  Y. Jeon [11], G. Leloudas [7], A. Melandri [6], R. Sanchez Ramirez [7],  A. de Ugarte Postigo [7], Y. Urata [9], D. Xu [12], S. Barthelmy [2],  A. Fruchter [13], N. Gehrels [2], N. Kawai [14], J. Norris [15],  C.C. Thoene [8], J. Racusin [2]}

\affiliation{[1] CRESST, [2] NASA/GSFC, [3] UMBC, [4] ORAU, [5] NAOJ, [6] INAF-OAB,  [7] Dark/NBI, [8] IAA-CSIC, [9] NCU, [10] ASIAA, [11] CEOU/SNU,  [12] WIS/NAOC, [13] STScI, [14] Titech, [15] BSU}

\bigskip

\noindent We present our successful program using Chandra for identifying the X-ray afterglow in sub-arcsecond accuracy for Swift/Fermi short GRB 111117A. Thanks to our rapid ToO request, Chandra clearly detected its X-ray afterglow.  No optical afterglow has been found in our deep optical observations by GMG, TNG, GTC and NOT.  Instead, we clearly detect the host galaxy in optical (GMG, TNG, GTC and NOT) and also in near-infrared (UKIRT and CFHT) bands.  We found that the best fit photometric redshift of the host is ~1.4. Furthermore, we clearly see the offset of 1.4\" between the host and the Chandra X-ray afterglow position.  We discuss the importance of using Chandra for obtaining a sub-arcsecond location of the afterglow in X-rays for short GRBs to study their host galaxies in great detail.
