% $ABSTRACT_SESSION_Vb_5b


\atitle{Energy injection in short GRBs and the role of magnetars}

\bigskip

\authors{A. Rowlinson [1], P.T. O'Brien [2]}

\affiliation{[1] University of Amsterdam, [2] University of Leicester}

\bigskip

\noindent A significant fraction of the Long Gamma-ray Bursts (GRBs) in the Swift sample show a plateau phase which may be due to ongoing energy injection. We find many Short GRBs detected by the Swift satellite show similar behavior. The remnant of NS-NS mergers may not collapse immediately to a BH (or even collapse at all) forming instead a magnetar. This model predicts that there would be a plateau phase in the X-ray lightcurve followed by a shallow decay phase, if it is a stable magnetar, or a steep decay if the magnetar collapses to a BH. By fitting this model to all of the Short GRB BAT-XRT lightcurves, we find that a significant fraction may show evidence of energy injection by a magnetar. This model can be tested using the next generations of gravitational wave observatories.
