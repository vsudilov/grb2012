% $POSTERS_SESSION_PII_p2
P-II-6

\atitle{On External Shock Model to Explain GeV Emission in GRB090926A}

\bigskip

\authors{Nissim Fraija, Rodrigo Sacahui, Magdalena Gonzalez and Willian Lee}

\affiliation{Astronomy Institute}

\bigskip

\noindent Previously, we have developed a SSC reverse shock model to describe the high-energy component lasting 2s in GRB980923. The model describes spectral indexes, fluxes and duration of the high-energy component as well as the long tail presented in GRB980923. Here, we present an extension of the model to describe the high-energy emission of GRB090926A. We first argue that the emission consist of two components, one with a duration less than 1s during the prompt phase and the second a long duration GeV emission lasting hundred of seconds after the prompt phase. The short high-energy emission is described as SSC emission from reverse shock similar to the observed in GRB980923. The long high-energy emission is described as SSC emission from afterglow forward shock.
