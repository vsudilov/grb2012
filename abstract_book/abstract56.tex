% $ABSTRACT_SESSION_VI_6


\atitle{Observing transients with LOFAR and AARTFAAC}

\bigskip

\authors{A. Rowlinson on behalf of the LOFAR Transients Key Project}

\affiliation{University of Amsterdam}

\bigskip

\noindent The Low Frequency Array (LOFAR) is a new European radio telescope operating at 30-240 MHz. LOFAR will offer an exciting opportunity to study the radio emission from GRBs at timescales from first detection until years afterwards. I will describe the features LOFAR will be able to use to study GRBs including:
1. Dedicated and commensal real-time monitoring programmes enabling detection and monitoring of radio afterglows.
2. Rapid, automated response to targets of opportunity.
3. A raw data buffer, providing the ability to \"re-observe\" past events at high time, spectral and spatial resolution.
4. Rapidly sharing newly detected transients using VOEvents.
Additionally, the AARTFAAC Project will build upon the core LOFAR capabilities to provide an all-sky real time monitor for bright transients.
