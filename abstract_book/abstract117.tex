% $POSTERS_SESSION_PIV_p4
P-IV-4

\atitle{A Significant Problem With Using the Amati Relation for cosmological Purposes}

\bigskip

\authors{Andrew C. Collazzi [1], Bradley E. Schaefer [2], Adam Goldstein [3], Robert D. Preece [3]}

\affiliation{[1] NASA/ORAU, [2] Louisiana State University, [3] The University of Alabama at Huntsville}

\bigskip

\noindent We study the distribution of many samples of Gamma-Ray Bursts when plotted in a diagram with their bolometric fluence versus the observed photon energy of peak spectral flux.  In this diagram, all bursts that obey the Amati relation must lie above some limiting line, although observational scatter is expected to be substantial.  While early bursts with spectroscopic redshifts are consistent with this Amati limit, we find that the bursts from BATSE, Swift, Suzaku, and Konus are all greatly in violation of the Amati limit, and this is true whether or not the bursts have measured spectroscopic redshifts.  This requires that selection effects are dominating these distributions, which we quantitatively identify.  As such, the Amati relation should not be used for cosmological purposes.

\index{\tiny{Collazzi, Andrew C.: \textit{A Significant Problem With Using the Amati Relation for cosmological Purposes}}}
