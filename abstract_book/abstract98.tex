% $POSTERS_SESSION_PII_p2
P-II-9

\atitle{The observation of Gamma Ray Bursts with AGILE}

\bigskip

\authors{E. Del Monte on behalf of the AGILE team}

\affiliation{INAF IAPS}

\bigskip

\noindent The AGILE satellite, launched in 2007, is continuing the observation of Gamma Ray Bursts (GRBs) with the two imagers,
SuperAGILE (18 - 60 keV) and the Gamma Ray Imaging Detector (20 MeV - 50 GeV), and the non-imaging Minicalorimeter (0.35 - 100 MeV). Up to now, about 0.5 GRBs per month are localised by SuperAGILE and around 1 GRB is detected per week by the Minicalorimeter. Four
bursts have been firmly detected by the Gamma Ray Imaging Detector while upper limits are provided for the non-detected events in the field of view. In this presentation we review the status of the observation of GRBs with AGILE, we show the properties of the most important events and we discuss the upper limits in the gamma-ray
band.

\index{\tiny{Del Monte, Ettore: \textit{The observation of Gamma Ray Bursts with AGILE}}}
