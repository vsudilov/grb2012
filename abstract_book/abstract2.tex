% $ABSTRACT_SESSION_I_1


\atitle{Scientific highlights from the Fermi Gamma-ray Burst Monitor}

\bigskip

\authors{Sheila McBreen on behalf of the GBM team}

\affiliation{University College Dublin}

\bigskip

\noindent The Fermi Gamma-ray Burst Monitor (GBM) is an all-sky instrument, sensitive in the 8 keV to 40 MeV energy range. GBM  detected over 900 Gamma-Ray Bursts (GRBs) in the four years since the launch of Fermi and the Large Area Telescope detected $\sim$8\% of those within the overlapping field of view. The spectral and temporal properties of the GBM GRB sample will be described. In addition to GRBs, GBM has triggered on other transient sources, such as soft gamma repeaters, terrestrial gamma-ray flashes and solar flares. I will discuss the performance, data products and scientific highlights of GBM.

\index{\tiny{McBreen, Sheila: \textit{Scientific highlights from the Fermi Gamma-ray Burst Monitor}}}
