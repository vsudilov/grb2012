% $POSTERS_SESSION_PIV_p4
P-IV-3

\atitle{The mystery of the missing GRB redshifts}

\bigskip

\authors{David Coward [1], Eric Howell [1], Tsvi Piran [2], Marica Branchesi [3], Dafne Guetta [4], Cadia Kannan [5]}

\affiliation{[1] School of Physics, University of Western Australia, [2] Racah Institute of Physics, The Hebrew University, [3] DiSBeF - Universita degli Studi di Urbino `Carlo Bo', [4] Department of Physics and Optical Engineering, ORT Braude, [5] Observatoire de la Cote dAzur}

\bigskip

\noindent Only about 30\% of long GRBs have measured redshifts. We constrain the dominant selection effects (both instrumental and astrophysical) to identify where the missing GRB redshifts are located. Surprisingly, we find that most of the missing bursts are located in z = 1-3, where the star formation rate is also peaking. We identify two main reasons why GRB redshifts are difficult to obtain in this region: the so-called redshift desert and dust extinction. We also find evidence for an evolution of optical brightness for high redshift bursts. It is actually relatively easier to acquire redshifts of high-z bursts, although the intrinsic rate of these bursts is small.

\index{\tiny{Coward, David: \textit{The mystery of the missing GRB redshifts}}}
