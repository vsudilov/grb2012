% $ABSTRACT_SESSION_VI_6


\atitle{Cosmic gamma-ray bursts studies with Ioffe Institute Konus experiments}

\bigskip

\authors{R.L. Aptekar, S.V. Golenetskii, D.D. Frederiks, E.P. Mazets, V.D. Pal'shin}

\affiliation{Ioffe Physico-Technical Institute, St. Peresburg, Russian Federation}

\bigskip

\noindent We present a short review of GRB studies being performed for many years by Ioffe Institute onboard a number of space missions. The first breakthrough in the studies of GRB was made possible by four Konus experiments carried out by the Ioffe Institute onboard the Venera 11 to 14 deep space missions in 1978 to 1983. A new important stage of our research  is associated with joint Russian-American experiment with the Russian Konus scientific instrument onboard the U.S. Wind spacecraft which has been successfully operating since its launch in November 1994. The Konus-Wind experiment has made an impressive number of important GRB observations and other astrophysical discoveries, due to the advantages of its design and its interplanetary locations. We also discuss our future GRB experiments.

\index{\tiny{Aptekar, Rafail: \textit{Cosmic gamma-ray bursts studies with Ioffe Institute Konus experiments}}}
