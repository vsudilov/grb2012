% $ABSTRACT_SESSION_IVb_4b


\atitle{The long gamma-ray burst rate and the correlation with host galaxy properties}

\bigskip

\authors{J. Elliott [1], J. Greiner [1], S. Khochfar [1], P. Schady [1], J. L. Johnson [1,2], A. Rau [1]}

\affiliation{[1] MPE Garching, [2] Los Alamnos Laboratory}

\bigskip

\noindent The association of long gamma-ray bursts (LGRB) with the death of massive stars allows the cosmic star formation history (CSFH) to be probed to higher redshifts than current conventional methods, however, no consensus on the manner in which the LGRB rate traces the CSFH has been reached. Driven by recent highly complete LGRB samples, obtained by GROND over the past 4 years, and new evidence of LGRBs occuring in more massive and metal rich galaxies than previously thought, the possible biases of the LGRBR-CSFH connection are investigated over a large range of galaxy properties. It is found that there is no strong preference for a metallicity cut or fixed galaxy mass boundaries and that there are no unknown redshift effects, in contrast to previous work.
