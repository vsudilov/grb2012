% $ABSTRACT_SESSION_Vb_5b


\atitle{Model of the extended emission of short gamma-ray bursts}

\bigskip

\authors{Alexei Pozanenko [1], Maxim Barkov [2,1]}

\affiliation{[1] Space Research Institute, [2] Max-Planck-Institut fur Kernphysik}

\bigskip

\noindent The existence of extended emission (EE) is an intriguing property of short-duration gamma-ray bursts: the nature of the EE is still unclear. We consider short gamma-ray bursts, emphasizing the common properties of both short bursts and short bursts with EE. We propose a two jet model which can describe both short main episode of hard spectra emission, specific for short bursts, and softer spectra EE by different off axis position of observer. The model involves a short-duration jet, which is powered by heating due to neutrino annihilation, and a long-lived Blandford–Znajek jet with a significantly narrow opening angle. The model is a plausible mechanism for short-duration burst energization. It can explain short bursts both with and without EE within a single class of progenitor
