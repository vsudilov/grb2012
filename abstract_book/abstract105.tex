% $POSTERS_SESSION_PIII_p3
P-III-15

\atitle{Late-time observations of the X-ray afterglow of GRB 060729}

\bigskip

\authors{D. Grupe [1], D. Burrows [1], X.F. Wu [2], B. Zhang [3], G. Garmire [1]}

\affiliation{[1] PSU,  [2] Nanjing Univ., [3] UNLV}

\bigskip

\noindent We summarize the results of the late-time Chandra observations of the X-ray afterglow of the Swift-discovered GRB 060729. These Chandra observations have been the latest X-ray detections of an afterglow even, up to 21 month after the trigger. The last two Chandra observations in December 2007 and May 2008 suggest a break at about a year after the burst, implying a jet half-opening angle of about 14 degrees, if interpreted as a jet break. As an alternative this break may have a spectral origin. In that case no jet break was observed and the half-opening angle is larger than 15 degrees for a wind medium. Comparing the X-ray afterglow of GRB 060729 with other bright X-ray afterglows we discuss why the afterglow of GRB 060729 was such an exceptionally long-lasting event.

\index{\tiny{Grupe, Dirk: \textit{Late-time observations of the X-ray afterglow of GRB 060729}}}
