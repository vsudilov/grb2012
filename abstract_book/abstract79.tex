% $POSTERS_SESSION_PII_p2
P-II-21

\atitle{The proto-black hole concept in GRB 101023 and its possible extension to GRB 110709B.}

\bigskip

\authors{A.V. Penacchioni [1]; R. Ruffini [2,3], C.L. Bianco [2,3], L. Izzo [2], M. Muccino [2], G. Pisani [1]}

\affiliation{[1] Erasmus Mundus Joint Doctorate IRAP PhD. Student, [2] Dip. di Fisica, Sapienza Università di Roma and ICRA,  [3] ICRANet}

\bigskip

\noindent A clear evidence of two components in GRB 101023 has been outlined. The first episode has been fit by a black body plus power-law spectral model. The temperature changes with time following a broken power-law, and the photon index presents a soft-to-hard evolution. The second episode appears to be a canonical GRB. Using the Amati and Atteia relations we determined the cosmological redshift to be $z \sim 0.9 \pm 0.08$ (stat.) $\pm 0.2$ (sys.). This source appears to be a twin of GRB 090618, and appear to be related to a single core collapse: the first episode is related to the proto-black hole and the second one to a canonical GRB. We are exploring the possibility to extend these considerations to GRB 110709B. GRB 110709B has been detected by Konus Wind and Swift in the high energy band, and

\index{\tiny{Penacchioni, Ana Virginia: \textit{The proto-black hole concept in GRB 101023 and its possible extension to GRB 110709B.}}}
