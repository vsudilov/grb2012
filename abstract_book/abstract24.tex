% $ABSTRACT_SESSION_IIIb_3b


\atitle{Observational Aspects of Gamma-ray Burst Afterglows}

\bigskip

\authors{Thomas Kruehler}

\affiliation{Dark Cosmology Centre}

\bigskip

\noindent For our understanding of the physics of GRB explosions as well as their implications on star formation and cosmology, it is crucial to observe the luminous afterglow emission. Afterglow observations provide an accurate distance scale to the GRB, and allow us to put constraints on the physical properties of the ultra-relativistic outflow. In the recent years, systematic follow-up campaigns have led to an increasing sample of high-quality afterglow light-curves and multi-wavelength SEDs. The new data sheds light on the properties of the population of long GRBs, including those events that have previously escaped detection.
I will give an overview of the state-of-the-art of the field and highlight new observational results.

\index{\tiny{Kruehler, Thomas: \textit{Observational Aspects of Gamma-ray Burst Afterglows}}}
