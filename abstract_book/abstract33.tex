% $ABSTRACT_SESSION_IVa_4a


\atitle{Difficulties in using GRBs as Standard Candles}

\bigskip

\authors{Adam Goldstein}

\affiliation{University of Alabama in Huntsville}

\bigskip

\noindent GRBs have been detected uniformly over the observable universe, ranging in comoving distance from a few hundred Mpc to a few thousand Mpc. For this reason there have been extensive studies into the possibility of using GRBs as standard candles. We discuss the attempts at defining GRBs as standard candles, such as the search for a robust luminosity indicator, pseudo-redshift predictions, the complications that emission collimation introduces into the estimation of the energetics, and the difficulty introduced by the widely varying observed properties of GRBs. These topics will be examined with data and analyses from both Fermi and Swift observations. Problems with current studies using GRBs as standard candles will be noted as well as potential paths forward to solve these problems.
