% $ABSTRACT_SESSION_IIb_2b


\atitle{The Photosphere in Gamma-Ray Bursts: Lessons Learned from Fermi}

\bigskip

\authors{Felix Ryde}

\affiliation{Royal Insitute of Technology}

\bigskip

\noindent In spite of extensive research over the past decades, a complete physical picture of the origin of the prompt GRB emission is still lacking. The failure of the synchrotron interpretation raises the need for alternatives.  During recent years, though, evidence has been  accumulating that the jet photosphere plays an important role.   I will summarise the lessons learned from Fermi observations regarding the behaviour of the photosphere.  I will concentrate on a few strong and important bursts, namely, GRB090902, GRB100724A, and GRB110721A.   In particular, I will show evidence for sub-photospheric energy dissipation, and show how it can explain the non-thermal spectra seen in many bursts.  Finally, I will discuss some new ideas that may enable us to extract more information from Fermi data.
