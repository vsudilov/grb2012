% $ABSTRACT_SESSION_IId_2d


\atitle{Broadband observations of GRB110731A with Fermi, Swift, GROND and MOA}

\bigskip

\authors{Johan Bregeon}

\affiliation{INFN Pisa}

\bigskip

\noindent We report on the multi-wavelength observations of the bright long GRB 110731A detected by Fermi and Swift, and follow-up observations by the MOA and GROND optical telescopes. The analysis of the prompt phase reveals that this GRB shares many features with the other bright bursts observed by Fermi-LAT during its first 3 years on-orbit. The wealth of multi-wavelength observations allowed
temporal and spectral analyses in different epochs, favoring emission from the forward shock in a wind-type medium. The temporally-extended emission observed by the LAT is most likely the highest-energy extension of the afterglow. Prompt emission from a single zone and forward shock afterglow analysis both lead independently to an estimated jet bulk Lorentz factor of $\Gamma$ $\sim$500.

\index{\tiny{Bregeon, Johan: \textit{Broadband observations of GRB110731A with Fermi, Swift, GROND and MOA}}}
