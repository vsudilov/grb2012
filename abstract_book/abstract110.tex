% $POSTERS_SESSION_PIII_p3
P-III-6

\atitle{A universal scaling for short and long gamma-ray bursts: E<sub>X,iso</sub>-E<sub>&gamma;,iso</sub>-E<sub>pk</sub>}

\bigskip

\authors{Maria Grazia Bernardini [1], Raffaella Margutti [2], Elena Zaninoni [1,3], Guido Chincarini [1]}

\affiliation{[1] INAF - Osservatorio Astronomico di Brera, [2] Harvard-Smithsonian Center for Astrophysics, [3] University of Padova, Physics & Astronomy Dept. Galileo Galilei}

\bigskip

\noindent The comprehensive statistical analysis of Swift X-ray light-curves, collecting data from six years of operation, revealed the existence of a universal scaling among the isotropic energy emitted in the rest frame $10-10^4$ keV energy band during the prompt emission ($E_{gamma,iso}$), the peak of the prompt emission energy spectrum ($E_{pk}$), and the X-ray energy emitted in the $0.3-10$ keV observed energy band ($E_{X,iso}$). We show that this three-parameter correlation is robust and does not depend on our definition of $E_{X,iso}$. It is shared by long, short, and low-energetic GRBs, differently from the well-known $E_{gamma,iso}-E_{pk}$ correlation. We speculate that the ultimate physical property that regulates the GRB properties is the outflow Lorentz factor.
