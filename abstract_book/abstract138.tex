% $POSTERS_SESSION_PVII_p7
P-VII-5

\atitle{Recent IceCube Results from Searches for Transient Neutrino Sources}

\bigskip

\authors{A. Homeier for the IceCube Collaboration }

\affiliation{Universität Bonn}

\bigskip

\noindent IceCube, a cubic kilometer neutrino detector located in glacial ice at the South Pole, has recently reached a sensitivity below the TeV-PeV neutrino flux that is predicted from gamma-ray bursts (GRBs) if those are responsible for the observed extragalactic cosmic-ray flux and are accelerating protons. No excess over background has been observed, which allows to constrain proton acceleration in the fireball model. Complementary an online neutrino multiplet selection allows IceCube to trigger Swift and a network of optical telescopes, which can then identify possible electromagnetic counterparts. This allows to probe for the connection between GRBs, SNe and relativistic jets in SNe. Results from the GRB search and a first limit on relativistic jets in SNe are presented.
