% $POSTERS_SESSION_PII_p2
P-II-8

\atitle{The IPN Supplement to the Fermi GBM 2-year Catalog}

\bigskip

\authors{K. Hurley [1], V. Pal'shin [2], M. Briggs, V. Connaughton, C. Meegan [3], S.Golenetskii, R. Aptekar, E. Mazets, D.  Frederiks [2], I. G. Mitrofanov, D. Golovin, M. L. Litvak, A. B.  Sanin [4], W. Boynton, C. Fellows, K.  Harshman, R. Starr [5], D. M. Smith [6], W.  Hajdas [7], A. Rau, X. Zhang, A. von Kienlin [8], K. Yamaoka [9], M.  Ohno, Y. Fukazawa [10], T. Takahashi [11], M., Tashiro [12], Y.  Terada [13], T. Murakami [14], K.  Makishima [15], S.  Barthelmy, J. Cummings, N.  Gehrels, H. Krimm, D. Palmer, T. Cline [16], J. Goldsten [17], E. Del Monte, M. Feroci [18], M. Marisaldi [19]}

\affiliation{[1] UC Berkeley Space Sciences Laboratory, [2] Ioffe Physico-Technical Institute, [3] CSPAR, UAH, [4] IKI, [5] Lunar and Planetary Laboratory, Univ. of Arizona, [6] UCSC SCIPP, [7] Paul Scherrer Institute, [8] MPI, [9] Aoyama Gakuin University, ...}

\bigskip

\noindent The Fermi GBM catalog contains localizations for 491 GRBs.  395 of them were observed by one or more spacecraft, in addition to Fermi, in the nine-spacecraft Interplanetary Network.  32 were observed by two interplanetary spacecraft, while 262 were observed either by a single interplanetary spacecraft or by Konus-Wind, and could be triangulated to small error boxes or annuli.  Triangulation reduces the GBM error box areas by up to four orders of magnitude.  We will present examples and statistics, and discuss how it facilitates projects such as helping the GBM team understand the systematics in their localization procedure, assisting the LAT team in searching for high energy emission associated with GRBs, and refining multi-wavelength and non-electromagnetic counterpart searches.
